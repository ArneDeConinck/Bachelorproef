%%=============================================================================
%% Conclusie
%%=============================================================================

\chapter{Conclusie}
\label{ch:conclusie}

% TODO: Trek een duidelijke conclusie, in de vorm van een antwoord op de
% onderzoeksvra(a)g(en). Wat was jouw bijdrage aan het onderzoeksdomein en
% hoe biedt dit meerwaarde aan het vakgebied/doelgroep? 
% Reflecteer kritisch over het resultaat. In Engelse teksten wordt deze sectie
% ``Discussion'' genoemd. Had je deze uitkomst verwacht? Zijn er zaken die nog
% niet duidelijk zijn?
% Heeft het onderzoek geleid tot nieuwe vragen die uitnodigen tot verder 
%onderzoek?

In deze studie hebben we onderzocht wat de verschillen zijn tussen een IT en OT-netwerk en waarom er wordt gekozen voor een convergentie. Tijdens het onderzoek werden bedrijven bevraagd over hun IT en OT-netwerk en wat ze vinden van de convergentie. Op basis van de informatie die verzameld is in de literatuurstudie en enquête kunnen we een conclusie formuleren op de hoofdonderzoeksvraag en deelonderzoeksvragen.

Dit onderzoek heeft aangetoond dat er een duidelijk verschil is tussen een IT en OT-netwerk en dat de convergentie van deze netwerken volop bezig is. We kunnen besluiten dat de componenten tussen de netwerken verschillen. Een IT-netwerk maakt meer gebruikt van servers en computers met als doel gegevens creëren, verwerken, beveiligen en uitwisselen, hierbij denken we aan een DHCP-server, DNS-server en dergelijke. Een OT-netwerk daarentegen maakt gebruik van componenenten met als doel fysieke apparaten, processen en gebeurtenissen te wijzigen, bewaken en controleren. Hierbij denken we aan plc's, SCADA en dergelijke. De protocollen tussen beide netwerken verschillen ook. We kunnen constateren dat de OT-protocollen meer en meer aan het veranderen zijn naar TCP/IP om een goede communicatie tussen beide netwerken te bezorgen. Beveiliging in beide netwerken wordt anders opgesteld. IT-beveiliging is in de meeste bedrijven heel goed, maar OT-beveiliging niet. Vroeger was het OT-netwerk afgescheiden en was hier geen tot weinig beveiliging nodig. Als bedrijven de convergentie toepassingen of hun OT-netwerk naar buiten openzetten moet dit ook goed beveiligd worden wat momenteel bij de meeste bedrijven nog niet het geval is. De convergentie van deze 2 netwerken is bij de meeste bedrijven volop bezig of staat op hun agenda. Uit de theorie en enquête blijkt dat bedrijven kiezen voor de convergentie voor de verschillende voordelen die er zijn. Zo zorgt de convergentie van beide netwerken voor een gegevensstroom van het OT-netwerk naar het IT-netwerk, zodat hier conclusies uit kunnen getrokken worden. Uit de enquête kunnen we afleiden dat er niet veel problemen optreden bij de convergentie. Nieuwe begrippen zorgen ervoor dat de convergentie plaatsvindt bij bedrijven. Door de internet of things wordt er veel meer data gegenereerd. Deze data kan van cruciaal belang zijn om conclusies te nemen.

Om een beter beeld te vormen, zou de enquête moeten worden uitgevoerd op een groter aantal bedrijven. Echter willen de meeste bedrijven niet veel vertellen over de beveiliging die ze gebruiken binnen hun omgevingen.



