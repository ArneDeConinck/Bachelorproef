%%=============================================================================
%% Samenvatting
%%=============================================================================

% TODO: De "abstract" of samenvatting is een kernachtige (~ 1 blz. voor een
% thesis) synthese van het document.
%
% Deze aspecten moeten zeker aan bod komen:
% - Context: waarom is dit werk belangrijk?
% - Nood: waarom moest dit onderzocht worden?
% - Taak: wat heb je precies gedaan?
% - Object: wat staat in dit document geschreven?
% - Resultaat: wat was het resultaat?
% - Conclusie: wat is/zijn de belangrijkste conclusie(s)?
% - Perspectief: blijven er nog vragen open die in de toekomst nog kunnen
%    onderzocht worden? Wat is een mogelijk vervolg voor jouw onderzoek?
%
% LET OP! Een samenvatting is GEEN voorwoord!

%%---------- Nederlandse samenvatting -----------------------------------------
%
% TODO: Als je je bachelorproef in het Engels schrijft, moet je eerst een
% Nederlandse samenvatting invoegen. Haal daarvoor onderstaande code uit
% commentaar.
% Wie zijn bachelorproef in het Nederlands schrijft, kan dit negeren, de inhoud
% wordt niet in het document ingevoegd.

\IfLanguageName{english}{%
\selectlanguage{dutch}
\chapter*{Samenvatting}
\lipsum[1-4]
\selectlanguage{english}
}{}

%%---------- Samenvatting -----------------------------------------------------
% De samenvatting in de hoofdtaal van het document

\chapter*{\IfLanguageName{dutch}{Samenvatting}{Abstract}}
% Deze aspecten moeten zeker aan bod komen:
% - Context: waarom is dit werk belangrijk? x
% - Nood: waarom moest dit onderzocht worden? x
% - Taak: wat heb je precies gedaan?x
% - Object: wat staat in dit document geschreven?x
% - Resultaat: wat was het resultaat?x
% - Conclusie: wat is/zijn de belangrijkste conclusie(s)?x
% - Perspectief: blijven er nog vragen open die in de toekomst nog kunnen
%    onderzocht worden? Wat is een mogelijk vervolg voor jouw onderzoek?
De IT en OT convergentie is een populair hedendaags onderwerp. Veel grote bedrijven zijn bezig met de convergentie om hun machines, applicaties en infrastructuur te optimaliseren. Echter is de convergentie niet simpel door middel van verschillende componenten, protocollen en beveiliging. De verschillen aankaarten tussen de twee netwerken zou bedrijven kunnen helpen bij hun convergentie. In dit onderzoek wordt enerzijds via een literatuurstudie onderzocht wat de verschillen zijn tussen beide netwerken en hoe het staat met de convergentie. Anderzijds wordt er aan de hand van een enquête onderzoek gedaan naar de IT en OT-omgeving en convergentie bij bedrijven. In deze bachelorproef staat geschreven wat de verschillen zijn tussen een IT en OT-netwerk en hoe het staat met de convergentie van beide netwerken. Uit het onderzoek blijkt dat de twee netwerken enkele verschillen hebben in componenten, protocollen en beveiliging. Deze verschillen worden doorheen de bachelorproef aangehaald. De convergentie is bij de meeste bedrijven aan de gang. Het onderzoek kan vervolgd worden via meerdere invalshoeken. De verschillen tussen beide netwerken qua protocollen en beveiliging zou elk jaar minder moeten worden, dit kan onderzocht worden. De convergentie tussen beide netwerken is een veelbesproken onderwerp momenteel. Veel bedrijven zijn hier momenteel mee bezig. Er kan nog extra onderzoek gevoerd worden naar hoe bedrijven dit hebben aangepakt binnen enkele jaren aangezien er dan weer nieuwe begrippen zullen opduiken.
