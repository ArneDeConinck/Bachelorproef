%%=============================================================================
%% Voorwoord
%%=============================================================================

\chapter*{\IfLanguageName{dutch}{Woord vooraf}{Preface}}
\label{ch:voorwoord}

%% TODO:
%% Het voorwoord is het enige deel van de bachelorproef waar je vanuit je
%% eigen standpunt (``ik-vorm'') mag schrijven. Je kan hier bv. motiveren
%% waarom jij het onderwerp wil bespreken.
%% Vergeet ook niet te bedanken wie je geholpen/gesteund/... heeft
Deze bachelorproef werd geschreven in het kader van het behalen van het diploma ``Bachelor in de Toegepaste Informatica'', afstudeerrichting Systeem- en Netwerkbeheer.

Gedurende de 3 academiejaren is netwerken veeluit aan bod gekomen in de opleiding Toegepaste Informatica. Hierbij zijn er verschillende netwerken aan bod gekomen met hun protocollen en manieren om deze op te zetten. Echter is de nadruk op het verschil tussen informationele en operationele netwerken nooit aan bod gekomen. Mijn interesse gaat verder dan enkel de theoretische kant hiervan, daarom zal er ook worden onderzocht hoever bedrijven staan met de convergentie van beide netwerken.

De scriptie zou niet tot stand zijn gekomen zonder enkele personen. Hierbij wil ik in het voorwoord graag gebruik maken om deze personen te bedanken.

