%%=============================================================================
%% Voorwoord
%%=============================================================================

\chapter*{\IfLanguageName{dutch}{Woord vooraf}{Preface}}
\label{ch:voorwoord}

%% TODO:
%% Het voorwoord is het enige deel van de bachelorproef waar je vanuit je
%% eigen standpunt (``ik-vorm'') mag schrijven. Je kan hier bv. motiveren
%% waarom jij het onderwerp wil bespreken.
%% Vergeet ook niet te bedanken wie je geholpen/gesteund/... heeft
Deze bachelorproef werd geschreven in het kader van het behalen van het diploma ``Bachelor in de Toegepaste Informatica'', afstudeerrichting Systeem- en Netwerkbeheer.

Gedurende de 3 academiejaren is netwerken uitgebreid aan bod gekomen in de opleiding Toegepaste Informatica. Hierbij zijn er verschillende netwerken aan bod gekomen met hun protocollen en manieren om deze op te zetten. Echter is de nadruk op het verschil tussen informationele en operationele netwerken nooit aan bod gekomen. Mijn interesse gaat verder dan alleen de theoretische kant hiervan, daarom wou ik ook onderzoeken hoever bedrijven staan met de convergentie van beide netwerken.

De scriptie zou niet tot stand zijn gekomen zonder enkele personen. Hierbij wil ik het voorwoord graag gebruiken om deze personen te bedanken. Als eerste wil ik Roobrouck Heidi bedanken. Dit was mijn bachelorproefbegeleider en heeft mij geholpen met alle vragen die ik had. Ik heb altijd zeer goede feedback gekregen waarvoor ik haar wil bedanken. Verder wil ik Vanderveken Luc bedanken. Dit was mijn bachelorproefcoördinator en heeft mij zo goed mogelijk proberen helpen met technische vragen. Ook heeft hij mij helpen zoeken naar bedrijven die mee wouden helpen aan dit onderzoek. Ook wil ik graag Rener Dylan en Casier Nick bedanken. Dit zijn 2 security consultants die werkten bij het bedrijf waar ik stage liep. Ze hebben mij altijd geholpen bij technische vragen die ik had en hebben mij veel zaken uitgelegd waar ik nog niet veel van wist. Als laatste wil ik graag mijn familie en vrienden bedanken. Ze hebben mij altijd geholpen en gesteund gedurende mijn hele schoolcarrière.

