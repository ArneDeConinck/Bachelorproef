%%=============================================================================
%% Inleiding
%%=============================================================================

\chapter{\IfLanguageName{dutch}{Inleiding}{Introduction}}
\label{ch:inleiding}
In het internet tijdperk waar we ons tegenwoordig in bevinden zijn netwerken niet meer uit ons leven weg te denken. Netwerken worden gebruikt om informatie te delen tussen twee of meerdere computers. Wereldwijd maakt iedereen hier gebruik van. Deze netwerken kunnen we in twee grote clusters verdelen, namelijk informationele en operationele netwerken. informationele en operationele netwerken werden aan het begin van hun bestaan volledig gescheiden opgesteld.~\footnote{Informationele technologie}IT staat in voor de waarborg van de vertrouwelijkheid en het delen van de informatie binnen een kantooromgeving.~\footnote{Operationele technologie}OT daarintegen staat in voor de beschikbaarheid en integriteit van informatie van industriële apparatuur, hierbij denken we bijvoorbeeld aan software die robots in een fabriek aanstuurt.


IT verandert voortdurend en elke dag komen er telkens nieuwe IT-concepten aan bod. Hierbij denken we aan de~\footnote{internet of things is het geheel aan apparaten die via internetverbindingen met andere apparaten of systemen in contact staan en daarmee gegevens uitwisselen}internet of things (IOT),~\footnote{Grote hoeveelheden data dat gebruikt wordt om onderlinge relaties bloot te leggen}big data en vele andere concepten. Deze nieuwe IT-concepten zorgen ervoor dat de vroegere afgescheiden IT en OT-netwerken dichter bij elkaar komen. Waar vroeger IT en OT volledig gescheiden waren, willen vele bedrijven deze aan elkaar koppelen om meer overzicht te krijgen en sneller te reageren als er iets fout loopt. De convergentie van beide netwerken is echter niet vanzelfsprekend~\autocite{Hayes2020}. Het kan zo zijn dat IT en OT verschillen in de hardware, protocollen en beveiliging die ze gebruiken, dan is het niet gemakkelijk om beide samen te integreren. Er zal onderzoek gedaan worden naar de verschillen tussen IT en OT-netwerken en of het al dan niet gemakkelijk is om deze netwerken te convergeren.

\section{\IfLanguageName{dutch}{Probleemstelling}{Problem Statement}}
\label{sec:probleemstelling}
Bedrijven die hun IT en OT-netwerk gescheiden hebben geïmplementeerd moeten zich de dag van vandaag de vraag durven stellen of dit nog altijd zo hoort. Vele nieuwe termen zoals big data en IOT maken het moeilijker om beide netwerken gescheiden te houden. De convergentie van beide netwerken zal ervoor zorgen dat de processen efficiënter kunnen verlopen en dat de globale kosten van het systeem zullen dalen. Naast de voordelen brengt de convergentie ook nadelen met zich mee. Zo wordt het OT-netwerk schadelijk voor alle soorten aanvallen dat de IT al jaren teistert, wat kan leiden tot datalekken, spionage en kaping.

Dit onderzoek zal uitmaken voor bedrijven die hun IT en OT gescheiden hebben of het nuttig is om deze te convergeren en welke voor- en nadelen dit met zich mee kan brengen.



\section{\IfLanguageName{dutch}{Onderzoeksvraag}{Research question}}
\label{sec:onderzoeksvraag}
\subsection{\IfLanguageName{dutch}{Hoofdonderzoeksvraag}{Head Research question}}
\label{subsec:hoofdonderzoeksvraag}
Zoals reeds aangehaald in sectie~\ref{sec:probleemstelling} zal dit onderzoek zich focussen op de verschillen tussen IT en OT-netwerken en of de convergentie voordelen heeft. Daaruit vloeit volgende hoofdonderzoeksvraag voort:
\begin{itemize}
    \item Wat zijn de verschillen tussen een IT en OT-netwerk en waarom wordt er gekozen voor een convergentie van beide?
\end{itemize}
Op deze onderzoeksvraag zal een theoretisch antwoord gegeven worden in hoofdstuk~\ref{ch:stand-van-zaken}. Naast de literatuurstudie gaan we ook onderzoeken wat de bedrijven hiervan vinden, het antwoord van de bedrijven is terug te vinden in hoofdstuk~\ref{ch:Resultaten}. Het antwoord zal ook terug te vinden zijn in hoofdstuk~\ref{ch:conclusie}.

\subsection{\IfLanguageName{dutch}{Deelonderzoeksvraag}{Deelonderzoeksvraag}}
\label{subsec:deelonderzoeksvraag}
Ter ondersteuning van de hoofdvraag die in sectie~\ref{subsec:hoofdonderzoeksvraag} werd opgesteld zijn er nog enkele bijhorende deelonderzoeksvragen opgesteld:
\begin{itemize}
    \item Hoe ziet een IT netwerk er uit: (protocollen, hardware, standaarden en beveiliging)?
    \item Hoe ziet een OT netwerk er uit: (protocollen, hardware, standaarden en beveiliging)?
    \item Zijn er verschillen tussen deze 2 netwerken?
    \item Wat is de meerwaarde van IT-OT convergentie?
    \item Hoever staan we met de IT-OT convergentie en wat zijn de problemen hierbij?  
    \item Is er een connectie tussen de convergentie en nieuwe IT-trends zoals de IOT?
    
\end{itemize} 
Op deze deelonderzoeksvragen zal gaandeweg door dit onderzoek een antwoord gegeven worden. Een samenvatting hiervan is te vinden in hoofdstuk~\ref{ch:conclusie}.
\section{\IfLanguageName{dutch}{Onderzoeksdoelstelling}{Research objective}}
\label{sec:onderzoeksdoelstelling}
Het hoofddoel van dit onderzoek is het beantwoorden van de hoofdonderzoeksvraag: wat zijn de  verschillen tussen een informationeel en operationeel netwerk en waarom wordt er gekozen voor een convergentie van beide. Een tweede doel van dit onderzoek is het onderzoeken hoe het gesteld staat met de omgevingen van hedendaagse bedrijven en of deze geconvergeerd zijn. Dit onderzoek is dan ook geslaagd als we op deze vragen een antwoord kunnen bieden.

\section{\IfLanguageName{dutch}{Opzet van deze bachelorproef}{Structure of this bachelor thesis}}
\label{sec:opzet-bachelorproef}
De rest van deze bachelorproef is als volgt opgebouwd:

In Hoofdstuk~\ref{ch:stand-van-zaken} wordt een overzicht gegeven van de stand van zaken binnen het onderzoeksdomein, op basis van een literatuurstudie.

In Hoofdstuk~\ref{ch:methodologie} wordt de methodologie toegelicht en worden de gebruikte onderzoekstechnieken besproken om een antwoord te kunnen formuleren op de onderzoeksvragen.

In Hoofdstuk~\ref{ch:Resultaten} worden de resultaten besproken van de bedrijven die deelnamen aan de enquête.

In Hoofdstuk~\ref{ch:conclusie}, tenslotte, wordt de conclusie gegeven en een antwoord geformuleerd op de onderzoeksvragen. Daarbij wordt ook een aanzet gegeven voor toekomstig onderzoek binnen dit domein.