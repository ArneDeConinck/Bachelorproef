%%=============================================================================
%% Inleiding
%%=============================================================================

\chapter{\IfLanguageName{dutch}{Inleiding}{Introduction}}
\label{ch:inleiding}

%%De inleiding moet de lezer net genoeg informatie verschaffen om het onderwerp te begrijpen en in te zien waarom de onderzoeksvraag de moeite waard is om te onderzoeken. In de inleiding ga je literatuurverwijzingen beperken, zodat de tekst vlot leesbaar blijft. Je kan de inleiding verder onderverdelen in secties als dit de tekst verduidelijkt. Zaken die aan bod kunnen komen in de inleiding~\autocite{Pollefliet2011}:

In het internet tijdperk waar we ons tegenwoordig in bevinden zijn netwerken niet meer uit ons leven te denken. Netwerken worden gebruikt om informatie te delen tussen 2 of meerdere computers. Wereldwijd maakt iedereen hier gebruik van. Deze netwerken kunnen we in 2 grote clusters verdelen, namelijk informationele netwerken en operationele netwerken. Informationele technologie (IT) en operationele technologie (OT) werden aan het begin van hun bestaan volledig gescheiden opgesteld. IT staat in voor de waarborg van de vertrouwelijkheid en het delen van de informatie binnen een kantooromgeving. OT daarintegen staat in voor de beschikbaarheid en integriteit van informatie van industriële apparatuur, hierbij denken we bijvoorbeeld aan software die robots in een fabriek aanstuurt.


IT verandert voordurend en elke dag komen er telkens nieuwe IT-concepten aan bod. Hierbij denken we aan de internet of things (IOT), big data,etc. Deze nieuwe IT-concepten zorgen ervoor dat de vroegere afgescheiden IT en OT netwerken dichter bij elkaar komen. Waar dat vroeger IT en OT volledig gescheiden waren willen vele bedrijven deze aan elkaar koppelen om meer overzicht te krijgen en sneller te reageren als er iets fout loopt. De convergentie van beide netwerken is echter niet vanzelfsprekend~\autocite{Hayes2020}. Het kan zo zijn dat IT en OT verschillende hardware, protocollen, standaarden en security gebruiken, dan is het niet gemakkelijk om beide samen te integreren. Er zal onderzoek gedaan worden naar de verschillen tussen IT en OT netwerken en of het al dan niet gemakkelijk is om deze netwerken te convergeren.




%%\begin{itemize}
%%  \item context, achtergrond 
%%  \item afbakenen van het onderwerp 
%%  \item verantwoording van het onderwerp, methodologie 
%%  \item probleemstelling
%%  \item onderzoeksdoelstelling
%%  \item onderzoeksvraag
%%  \item \ldots
%%\end{itemize}

\section{\IfLanguageName{dutch}{Probleemstelling}{Problem Statement}}
\label{sec:probleemstelling}

%%Uit je probleemstelling moet duidelijk zijn dat je onderzoek een meerwaarde heeft voor een concrete doelgroep. De doelgroep moet goed gedefinieerd en afgelijnd zijn. Doelgroepen als ``bedrijven,'' ``KMO's,'' systeembeheerders, enz.~zijn nog te vaag. Als je een lijstje kan maken van de personen/organisaties die een meerwaarde zullen vinden in deze bachelorproef (dit is eigenlijk je steekproefkader), dan is dat een indicatie dat de doelgroep goed gedefinieerd is. Dit kan een enkel bedrijf zijn of zelfs één persoon (je co-promotor/opdrachtgever).

Bedrijven die hun IT en OT netwerk gescheiden hebben geïmplementeerd moeten zich de dag van vandaag de vraag durven stellen of dit nog altijd zo hoort. Vele nieuwe termen zoals big data en IOT maken het moeilijker om beide netwerken gescheiden te houden. De convergentie van beide netwerken zal ervoor zorgen dat de processen efficiënter kunnen verlopen en dat de globale kosten van het systeem zullen dalen. Naast de voordelen brengt de convergentie ook nadelen met zich mee. Zo wordt het OT netwerk schadelijk voor alle soorten aanvallen dat de IT al jaren teisters wat kan leiden tot datalekken, spionage en kaping.

Dit onderzoek zal uitmaken voor bedrijven die hun IT en OT gescheiden hebben of het nuttig is om deze te convergeren en welke voor en nadelen dit met zich mee kan brengen.



\section{\IfLanguageName{dutch}{Onderzoeksvraag}{Research question}}
\label{sec:onderzoeksvraag}
%%Wees zo concreet mogelijk bij het formuleren van je onderzoeksvraag. Een onderzoeksvraag is trouwens iets waar nog niemand op dit moment een antwoord heeft (voor zover je kan nagaan). Het opzoeken van bestaande informatie (bv. ``welke tools bestaan er voor deze toepassing?'') is dus geen onderzoeksvraag. Je kan de onderzoeksvraag verder specifiëren in deelvragen. Bv.~als je onderzoek gaat over performantiemetingen, dan 
\subsection{\IfLanguageName{dutch}{Hoofdonderzoeksvraag}{Head Research question}}
\label{subsec:hoofdonderzoeksvraag}
Zoals reeds aangehaald in sectie~\ref{sec:probleemstelling} zal dit onderzoek zich focussen op de verschillen tussen IT en OT netwerken en of deze makkelijk te convergeren zijn. Daaruit vloeit volgende hoofdonderzoeksvraag voort:
\begin{itemize}
    \item Wat zijn de verschillen tussen en IT en OT netwerk en waarom wordt er gekozen voor een convergentie van beide?
\end{itemize}

\subsection{\IfLanguageName{dutch}{Deeldonderzoeksvraag}{Head Research question}}
\label{subsec:deeldonderzoeksvraag}
Ter ondersteuning van de hoofdvraag die in sectie~\ref{subsec:hoofdonderzoeksvraag} werd opgesteld zijn er nog enkele bijhorende deelonderzoeksvragen opgesteld:
\begin{itemize}
    \item Hoe ziet een IT netwerk er uit: (protocollen, hardware, standaarden en security)?
    \item Hoe ziet een OT netwerk er uit: (protocollen, hardware, standaarden en security)?
    \item Zijn er verschillen tussen deze 2 netwerken?
    \item Wat is de meerwaarde van IT-OT convergentie?
    \item Hoever staan we met de IT-OT convergentie en wat zijn de problemen hierbij?  
    \item Is er een connectie met de convergentie en nieuwe IT trends zoals de IOT?
    
\end{itemize} 

\section{\IfLanguageName{dutch}{Onderzoeksdoelstelling}{Research objective}}
\label{sec:onderzoeksdoelstelling}

Wat is het beoogde resultaat van je bachelorproef? Wat zijn de criteria voor succes? Beschrijf die zo concreet mogelijk. Gaat het bv. om een proof-of-concept, een prototype, een verslag met aanbevelingen, een vergelijkende studie, enz.

\section{\IfLanguageName{dutch}{Opzet van deze bachelorproef}{Structure of this bachelor thesis}}
\label{sec:opzet-bachelorproef}

% Het is gebruikelijk aan het einde van de inleiding een overzicht te
% geven van de opbouw van de rest van de tekst. Deze sectie bevat al een aanzet
% die je kan aanvullen/aanpassen in functie van je eigen tekst.

De rest van deze bachelorproef is als volgt opgebouwd:

In Hoofdstuk~\ref{ch:stand-van-zaken} wordt een overzicht gegeven van de stand van zaken binnen het onderzoeksdomein, op basis van een literatuurstudie.

In Hoofdstuk~\ref{ch:methodologie} wordt de methodologie toegelicht en worden de gebruikte onderzoekstechnieken besproken om een antwoord te kunnen formuleren op de onderzoeksvragen.

% TODO: Vul hier aan voor je eigen hoofstukken, één of twee zinnen per hoofdstuk

In Hoofdstuk~\ref{ch:conclusie}, tenslotte, wordt de conclusie gegeven en een antwoord geformuleerd op de onderzoeksvragen. Daarbij wordt ook een aanzet gegeven voor toekomstig onderzoek binnen dit domein.